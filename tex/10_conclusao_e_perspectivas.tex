% !TEX root = 00_tcc.tex
\clearpage

\section{Conclusão e Perspectivas}

Como visto em, o processo de aplicação de um sistema híbrido é uma problemática
com diversos fatores que a influenciam, desde a idealização até operação.  As
fases de configuração, dimensionamento e operação passam todas por processos de
otimização, cada uma com particularidades.  As métricas abordadas auxiliaram em
decisões de desenho melhores.

A utilização da mesma arquitetura para todas \acrshort{ann} pode ter sido uma
das causas, visto que os fenômenos meteorológicos tem frequências diferentes,
como início e fim do dia ou estações do ano. Melhorias podem ser feitas através
da decomposição dos sinais, como no estudo de~\cite{Liu_2018}, citado
anteriormente.  Para as redes da irradiação solar, em estudos futuros é possível
conseguir melhores resultados realizando treinamento em dados apenas do período
diurno. Para determinar momentos em que há sol, o ângulo azimutal pode ser
indicador, considerando limites de nascer e pôr do sol. Uma função custo
diferente, como a correlação, para o treinamento poderia minimizar a diferença
de amplitude de sinal causada com a \acrshort{rmse}.

Os resultados expuseram que,  quando aplicada em contextos mais simples, as
estratégia se assemelham. A aplicação pode ser estudada com maior granularidade
nos critérios para uma avaliação mais diversa.
