% !TEX root = 00_tcc.tex
\clearpage

\section*{\centering Resumo}

Um sistema híbrido é composto por múltiplas fontes de energia.
Aquelas que possuem dependência meteorológica constituem incerteza no suprimento
da demanda, podendo comprometer o funcionamento dos serviços abastecidos. Por
isso, o operador do sistema precisa estar apto a prever o balanço energético e
decidir a melhor estratégia de controle.

A previsão de diversas séries temporais é uma tarefa difícil de ser feita
analiticamente por isso métodos envolvendo inteligência artificial tem sido
abordados. Dados históricos são necessários para serem processados mas os
modelos possuem grande capacidade de generalização, adequando-se ao cenário.

O presente trabalho teve como objetivo estudar a viabilidade de análise
preditiva em sistemas híbridos utilizando aprendizado de máquina. As fontes
abordadas foram eólica, solar e diesel, juntamente com reserva de banco de
baterias.  Os dados utilizados foram retirados da rede \acrshort{sonda}, fornecidos pelo
\acrshort{inpe}, na estação meteorológica de Petrolina.

Foi utilizado aprendizado supervisionado através de rede neurais. O algoritmo
aplicado foi uma \emph{Recurrent Neural Network}. Foi usada a linguagem
Python com auxílio de bibliotecas como \emph{numpy}, \emph{sklearn} e \emph{tensorflow}.

As redes neurais apresentaram bom desempenho em horizontes de curto prazo mas
degeneraram com passar dos timesteps. A estratégia preditiva teve acurácia de
73\% e foi capaz de obter resultados similares às tradicionais. Ainda há espaço
para melhora em trabalhos futuros.

\vfill
\textbf{Palavras-chave:} Série Temporal, Aprendizado de Máquina, Sistemas
Híbridos, Redes Neurais
\vfill
