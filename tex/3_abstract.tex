% !TEX root = 0_tcc.tex
\clearpage

\section*{\centering Abstract}


A hybrid system is composed of multiple energy sources. Those with weather
dependency represent uncertainty in demand provision, which could compromise the
functioning of the supplied services. Therefore, the system operator needs to be
able to predict the energy balance and decide the best control strategy.

Predicting multiple time series is a difficult task to be done analytically,
hence artificial intelligence has been used as alternative. Historical data has
to be processed but the models have great generalization capacity, adapting to
the scenario.

The present work aimed to study the feasibility of predictive analysis in hybrid
systems using machine learning. Wind, solar and diesel sources were addressed
along with energy storage. The data used was taken from the \acrshort{sonda} network,
provided by \acrshort{inpe}, at the Petrolina meteorological station.

Neural Network was used as supervised learning. The algorithm applied was the
\emph{Recurrent Neural Network}. Python language was used with the help of
libraries like \emph{numpy}, \emph{sklearn} and \emph{tensorflow}.

Neural networks performed well in short-term horizons but
degraded with longer timesteps. The predictive strategy had accuracy of
73\% and was able to obtain similar results to traditional ones. There is still space
for improvement in future work.

\vfill
\textbf{Keywords:} Time Series, Machine Learning, Hybrid Systems, Neural
Networks
\vfill
