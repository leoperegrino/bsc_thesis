% !TEX root = 0_tcc.tex
\clearpage

\section{Introdução}

\subsection{Eletricidade no Meio Urbano e Rural}

Atualmente, o perímetro urbano costuma ter disponibilidade elétrica de forma
ininterrupta. Entretanto, a medida em que afasta-se dos centros urbanos,
o acesso à energia se torna cada vez mais difícil. Em 2018, 668 milhões de
pessoas vivem no campo sem acesso à eletricidade, de acordo com~\cite[pág. 4]{irena:sdg7}.
A disparidade pode ser percebida quando vista por imagens de
satélite, como na Figura~\ref{fig:terra}.

\begin{figure}[h]
    \centering
    \includegraphics[width=0.8\textwidth]{../img/nasa.jpg}
    \caption{Visualização da eletrificação no mundo. Fonte:~\cite{nasa:earth}}\label{fig:terra}
\end{figure}

De forma a contornar tais limitações é comum a prática de geração de energia
localmente. A aplicação pode ser feita de forma isolada --- nenhuma conexão com
rede externa é feita --- ou com auxílio parcial da rede.  Quando feita
desconectada, também conhecida como \emph{off-grid}, é comum em localizações
remotas com difícil acesso à linhas de transmissão. Atualmente no Brasil,
existem 212 localidades isoladas, a maioria na região Norte segundo~\cite{ons:iso},
visto que é a mais densamente vegetada.

Também pode ser feito o abastecimento parcial do consumo.  No Brasil, a
resolução 786 institui a geração distribuída, com visto
em~\cite{aneel:ren482}.  De acordo com a resolução, é possível converter geração
excedente em créditos na concessionária respectiva.  Dessa forma, é possível
tornar rentável aplicações menores que não fazem uso de baterias.  De acordo com
o Balanço Energético Nacional em~\cite{epe:ben2020}, mini e micro geração
distribuída tiveram um crescimento de 169\% em relação ao ano anterior.

\subsection{Sistemas Híbridos}

Para tornar-se sustentável fora da rede convencional, \emph{off-grid}, diversas
fontes podem ser usadas em conjunto. Um grupo de tais fontes constitui um
sistema híbrido. As suas aplicações vão além de residenciais, sendo usadas
também para serviços de baixo custo de operação e manutenção. Sistemas
eletrônicos embarcados como sinalização de trânsito, iluminação pública ou
pequenas redes de rádio-telefonia; bombeamento e dessalinização de água por
parte de moradores de zonas pobres são alguns exemplos como mostrado
em~\cite[cap. 1.4]{kaldellis_2010}.

Sistemas híbridos são comumente compostos por fonte eólica e solar e dependem de
questões meteorológicas sendo, portanto, inadequados para encarregar-se de
manter oferta constante de serviços como hospitais.  A diversificação é usada
para mitigar possíveis intermitências no fornecimento, ao aproveitar a
complementariedade entre as fontes.  Como resguardo, as deficiências são
contornadas com banco de baterias ou outra fonte estável, feito diesel.

Fontes intermitentes, isto é, variam no tempo, são séries temporais com vários
parâmetros.  Como o conjunto, além da complexidade, possui extrema importância para o
abastecimento elétrico, faz-se necessário modelar e prever seu comportamento,
mitigando possíveis interrupções.

\subsection{Inteligência Artificial}

Na última década, pôde-se perceber um crescimento acentuado do uso de
inteligência artificial. Em especial, as chamadas redes neurais artificiais,
\acrlong{ann} (\acrshort{ann}), foi um dos métodos que mais se
destacaram.  O modelo possui esse nome pois assemelha-se ao cérebro humano,
composto por inúmeros neurônios interconectados capazes de aprender a partir do
ambiente.  O algoritmo baseia-se no classificador Perceptron
de~\cite{rosenblatt1958perceptron}.

\acrshort{ann} por ser computacionalmente custosa é dependente do desenvolvimento de
máquinas capazes de realizar. Recentemente, com a expansão da computação em
nuvem, \emph{cloud computing}, e grandes \emph{data centers}, \acrshort{ann}
tornaram-se acessíveis.  O algoritmo tem alta capacidade de generalização e, por
isso, é usado em situações de alta complexidade, como: preços de ações,
meteorologia, demanda por energia, entre outros.
A desvantagem de métodos baseados em redes neurais é devido à dificuldade de
interpretação sobre os parâmetros aprendidos pelo algoritmo.

\subsection{Apresentação e Objetivos}

As métricas para atuar no controle do sistema requer a avaliação de possíveis estados
futuros. As sobrecargas e déficits precisam ser manejadas com antecedência devido à
inércia mecânica do sistema diesel ou indisponibilidade do banco de bateria. Por
isso, análise preditiva é utilizada para mitigar essas incertezas.

Inteligência artificial tem adquirido relevância na área de geração de
energia, com diversos artigos lançados sobre o tema recentemente.

O trabalho propõe como objetivo geral estudar o uso de redes neurais na previsão
de múltiplas séries temporais: eólica e solar. Através disso, investigar
a aplicabilidade de análise preditiva para estratégias de controle de
um sistema híbrido.
