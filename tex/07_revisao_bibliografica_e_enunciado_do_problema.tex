% !TEX root = 00_tcc.tex
\clearpage

\section{Revisão Bibliográfica e Enunciado do Problema}

Uma dos maiores dificuldades de implementação de um sistema híbrido de energia é
a insegurança de abastecimento. Como as fontes renováveis dependem de efeitos
meteorológicos, é comum o super dimensionamento da aplicação.  Tal prática
resulta em um desenho de sistema custoso mas com vista a garantir a
disponibilidade elétrica, como visto em~\cite{Deshmukh_2008}.

Ainda em~\cite{Deshmukh_2008}, é visto que abordagem mais frequente para estimar um sistema
híbrido é avaliar cada fonte separadamente para depois as analisar em conjunto.
Caso as previsões individuais sejam acuradas o suficiente, a previsão conjunta
será a ótima.

~\cite[cap. 3]{kaldellis_2010} propõe que o problema de desenho de um sistema
híbrido ideal pode ser dividido em 4 etapas: síntese, design, operação e
otimização. A Tabela~\ref{tbl:metodo} sintetiza cada uma das etapas propostas. A
otimização pode ser aplicada em todas as outras etapas de forma a melhorar o
resultado. O problema de operação é crítico, dado que existem muitos modos
alternativos de operar um sistema e satisfazendo diferentes métricas
operacionais.

% !TEX root = ../00_tcc.tex

\begin{table}[ht]
 \centering
 \caption{Metodologia de dimensionamento de um sistema híbrido}\label{tbl:metodo}
  \begin{tabular}{l l l}
   \hline
   Problema  & Entrada/dado  & Resultado          \\
   \hline
   \hline
   Síntese/                &                                                 &                                         \\
   Configuração            & Requerimentos do sistema.                       &       Componentes       incluídos       \\
                           & Disponibilidade de recurso.                     & Topologia                               \\
                           & Considerações alternativas.                     &                                         \\
                           & Especificações básicas.                         &                                         \\
   Design/                 &                                                 &                                         \\
   Dimensionamento         & Configuração do sistema                         & Tamanho       dos      componentes      \\
                           & Resultados do problema de síntese.              &  Custo de investimento                  \\
                           & Requisitos detalhados do sistema.               &                                         \\
                           & Dados básicos                                   &                                         \\
                           &                                                 &                                         \\
   Análise de operação     & Projeto de sistema.                             & Cálculo de fluxos,                      \\
                           & Modo operacional.                               & eficiências, custos, etc                \\
                           & Requisitos, dados técnicos.                     &                                         \\
                           & Dados de custo.                                 &                                         \\
                           &                                                 &                                         \\
   Otimização              & Critérios a serem otimizados.                   &         solução       para o problema,  \\
                           & Restrições operacionais                         & cumpridas as restrições                 \\
                           & Restrições               técnicas               &                                         \\
                           & Restrições                          ambientais  &                                         \\
          \hline
  \end{tabular}
\end{table}


Como visto em~\cite{Kusakana_2013} existem vários métodos de dimensionamento de
sistemas híbridos solar-eólico tais como: suprir a média mensal anual;
considerar o mês mais desfavorável; evitar a probabilidade de perda no
fornecimento de energia (\acrshort{lpsp}); métodos de dimensionamento usando
software; entre outros.

Em~\cite{Deshmukh_2008} também é feita uma revisão bibliográfica sobre a
modelagem de componentes, desenho e métricas de \acrlong{hres}
(\acrshort{hres}).  Foi observado que aproximadamente 90\% dos estudos
realizados são sobre aspectos de modelagem ou econômicos.  Em relação às fontes
presentes, energia fotovoltaica com eólica representam 56\% do número de
publicações, seguido por \acrshort{hres} de fotovoltaica com 23\% em seguida de
eólica com 21\%.  Outras combinações híbridas revisadas em fase de pesquisa
incluem hidro, biomassa, célula de combustível e lixo municipal.

O consumo de combustível é um dos componentes mais onerosos da vida útil de um
gerador a diesel.  Portanto, determinar o melhor momento para iniciar e parar o
diesel é um fator crucial para a otimização.  De acordo
com~\cite{ashari1999optimum}, essas operações são geralmente feitas com base em
certa porcentagem de demanda do sistema ou o estado de carga da bateria
(\acrshort{soc}).  No estudo, para determinar o valor ótimo de arranque do
diesel, realizou-se a comparação dos custos operacionais do gerador diesel e da
bateria.  O autor concluiu que operar o sistema em uma estratégia que considere
o custo de uso da bateria, o custo de consumo do combustível diesel e o perfil
de carga fornece um menor custo de operação.

No estudo paramétrico de~\cite{elhadidy2000parametric}, ficou evidente que a
geração de energia a diesel é consideravelmente menor com inclusão banco de
baterias. Verificou-se que, em média, com a presença de 3 dias de autonomia da
bateria, cerca de 27\% da carga anual é alimentada a partir do sistema diesel.
Com a eliminação do armazenamento da bateria, cerca de 48\% da carga anual
precisa ser fornecido pelo sistema diesel.

~\cite{gupta2011modelling} propõe um algoritmo de controle para operação
otimizada de com estratégias de controle combinadas.  Cinco estratégias podem
ser usadas, são elas: carga da bateria, descarga da bateria, \acrlong{lf},
\acrlong{cc} e Peak Shaving. A Tabela~\ref{tbl:dispatch} descreve cada uma das
estratégias. O algoritmo apresentado foi capaz de projetar com eficiência um
sistema de eletrificação ótimo. Apesar das flutuações de radiação solar, o
gerador a diesel é capaz de manter a potência constante.

% !TEX root = ../00_tcc.tex

\begin{table}[ht]
	\centering

    \begin{tabular}{l p{10cm}}
		\hline
		Estratégia    &   Caracterização \\
		\hline
		\hline
        Carregamento da Bateria &
        A bateria carrega até o \acrshort{soc} máximo for atingido.
        \\
        \\
        Descarregamento da Bateria &
        A bateria descarrega até um dos casos:

        O \acrshort{soc} mínimo para descarga for atingido.

        A energia renovável é suficiente para atender a carga.

        A carga líquida é igual ou maior que a potência mínima de operação do diesel.
        \\
        \\
        \acrlong{lf} &
        O gerador a diesel funciona para seguir a carga líquida, sem carga ou descarga para a bateria.

        Um tempo mínimo de execução de diesel também é aplicado para evitar frequência alta de partida/parada.
        \\
        \\
        \acrlong{cc} &
        O gerador a diesel funciona para cobrir a carga líquida e carregar a bateria. O diesel continua durante o tempo mínimo de execução definido; depois disso, o o diesel continua funcionando até que uma das condições seja atendida:

        O \acrshort{soc} desejado foi atingido.

        A energia renovável é suficiente para atender a carga, carregando ou não
        as baterias.
        \\
        \\
        \emph{Peak Shaving} &
        O diesel operara com potência total.

        A Bateria é usada apenas para atender às flutuações momentâneas em torno da carga líquida \\
        \hline
	\end{tabular}

	\caption{Estratégias de controle combinadas}\label{tbl:dispatch}
\end{table}


Segundo~\cite{homermanual}, a estratégia de \acrlong{cc} consiste em operar o
gerador com toda potência disponível e escoar a energia líquida para cargas de
menor prioridade, como baterias.  A estratégia de \acrlong{lf} consiste em
colocar o gerador diesel para apenas acompanhar a carga primária, apenas
produzindo o suficiente para o abastecimento. As cargas de menor prioridade
nessa situação ficam a cargo das fontes renováveis.

Em~\cite{tazvinga2014energy} é empregado o \acrlong{mpc}\footnote{Modelo de
Controle Preditivo} (\acrshort{mpc}), um processo de controle que satisfaz um
grupo de restrições do sistema usando uma função de custo definida
explicitamente pelo usuário.  O trabalho faz uma comparação entre modelos de
\emph{loop} fechados e abertos.  O modelo de \emph{loop} aberto não possuem um
mecanismo de \emph{feedback} entre as predições de cada \emph{timesteps}, ou
seja: $\hat{y}_{t}$ e $\hat{y}_{t+1}$ não são relacionados.  A ausência de
\emph{feedback} pode tornar o sistema vulnerável a perturbações nas entradas.

Um \acrshort{mpc} em \emph{loop} fechado é proposto para
um sistema híbrido solar-eólica-diesel-bateria, com as seguintes restrições:
\begin{itemize}
	\item demanda de carga em cada momento é satisfeita
	\item energia fornecida pelo gerador diesel é minimizada
	\item o sistema de \emph{loop} fechado é robusto com relação a distúrbios na demanda de carga e saída de energia renovável
\end{itemize}

Duas simulações foram feitas em cada modelo de \emph{loop}, uma com perturbações
outra sem. No cenário sem perturbações, a performance dos dois modelos é muito
similar, o consumo de diesel é aprochadamente igual. No outro cenário é suposto
que o sistema encontra um condição ruim: a demanda de carga é 20\% maior que o
esperado e a energia eólica e solar são, cada uma, 20\% menor que o esperado.
Foi percebido que o desempenho do sistema de \emph{loop} fechado é geralmente
melhor, indicando que sua robustez com relação a distúrbios é superior ao
sistema de \emph{loop} aberto.  A razão é que o MPC é capaz de prever estados
futuros com base em \emph{feedback} dos estados atuais, influenciados por
distúrbios.  Em contraste, o controle de \emph{loop} aberto é incapaz de
responder a perturbações imprevisíveis e simplesmente começa o gerador diesel
quando a demanda de carga é maior do que o esperado Embora \acrshort{mpc} pode
ser sofisticada para aplicações domésticas individuais, ainda pode ser benéfico
para aplicações industriais.

De forma a automatizar o processo de dimensionamento, alguns softwares foram
desenvolvidos, como \acrshort{homer}, IHOGA e o Hybrid2.~\cite{Upadhyay_2014}
faz um comparativo de entradas e saídas de cada programa. A
Tabela~\ref{tbl:soft} apresenta as diferenças verificadas.

% !TEX root = ../0_tcc.tex

\begin{table}[ht]
	\centering
	\caption{Softwares disponíveis}\label{tbl:soft}
	\begin{tabular}{lll}
		\hline
        Software & Entrada & Saída \\
		\hline
		\hline
		\acrshort{homer} &   Demanda                     & Dimensionamento ideal                   \\
		                 &   Recursos                    & \acrshort{coe}                          \\
		                 &   Detalhes dos componentes    & \acrshort{npv}                          \\
                         &   Controle do sistema         & Percentual renovável                    \\
                         &                               &                                         \\
        HYBRID2          &   Demanda                     &  Dimensionamento com otimização         \\
		                 &   Recursos                    &  Emissão percentual de gases poluentes  \\
		                 &   Investimento inicial        &  \acrshort{coe}                         \\
						 &   Detalhes dos componentes    &  Payback                                \\
                         &                               &                                         \\
        RET Screen       &   Demanda                     &  Custos                                 \\
		                 &   Tamanho do array solar      &  Redução de emissão                     \\
		                 &   Database climática          &  Viabilidade financeira                 \\
		                 &   Detalhes dos componentes    &  Análise de risco                       \\
                         &                               &                                         \\
        IHOGA            &   Demanda                     &  Otimização multi-objetivo              \\
						 &   Recursos                    &  \acrshort{coe}                         \\
		                 &   Detalhes dos componentes    &  Emissão na vida útil                   \\
                         &                               &  Análise de compra e venda de energia   \\
		\hline
	\end{tabular}
\end{table}


Em matéria de inteligência artificial,~\cite{Upadhyay_2014} sumariza métodos
usados na literatura para o dimensionamento de um \acrshort{hres}. Todos estudos
abordaram a partir de uma perspectiva econômica, com otimização de indicadores
como \acrshort{lce}, custo de operação total, custo total do sistema ou custo de
energia. As entradas de cada modelo limitam-se majoritariamente a dados
periódicos de radiação solar e velocidade do vento. Segundo o autor, essa
abordagem através de \acrshort{ia} pode ser programada para convergir para a
melhor solução mas pode tornar-se ineficiente com o aumento de parâmetros do
sistema.

Em redes neurais, a abordagem é altamente dependente dos dados a serem
processados. Além de ser necessário quantidade, a qualidade dos dados é
determinante para as previsões. Séries temporais possuem características que,
quando não tratadas previamente, podem enviesar o resultado.  Apesar dos modelos
serem muito versáteis e capazes de aprender padrões, ainda sim é benéfico
realizar tratamentos para melhorar a performance de previsões temporais, de acordo
com~\cite{Zhang_2005}.  As estações do ano representam variações em intervalos
de tempo fixos, chamada sazonalidade. Possuir dados durante todo o ano permite o
modelo aprender esse padrão. As mudanças climáticas representam uma tendência de
crescimento e decrescimento constante em alguns parâmetros.

A literatura de \acrshort{ia} aplicada a fontes renováveis é mais extensa em
cada fonte individualmente, quando comparada à disponível em \acrshort{hres}.
Em~\cite{Liu_2018} é avaliada a predição de velocidade do vento através de redes
neurais \emph{Elman} e \acrshort{lstm}. O trabalho apresentou a decomposição do
sinal do vento em frequências diferentes através da transformada empírica
\emph{wavelet}. A \acrshort{ewt} é composta de filtros de banda da transformada
de Fourier, com vista a separar padrões e comportamento estocástico. A
combinação desses algoritmos mostrou resultados satisfatórios para previsão de
múltiplos intervalos de tempo.

~\cite{elsheikh2019modeling} realizou uma revisão de trabalhos usados para a
modelagem solar através de redes neurais. Aplicações de \acrshort{ann} em
energia solar foram feitas em áreas como térmica, fotovoltaicas e
concentradores. Os benefícios incluem capacidade de generalização, evitar
resolver modelos matemáticos complexos, reduzir tempo gasto em modelagem e
reduzir gastos com experimentos.

\subsection{Enunciado}

Como visto, conciliar as particularidades de cada fonte em um sistema híbrido é
uma tarefa sujeita a diversos fatores. Mesmo após o dimensionamento da aplicação
ainda é necessária a operação adequada para manutenção de expectativa de vida
dos componentes e para suprir a demanda.

A problemática a ser abordada em seguida será centrada em um consumidor com
necessidade de alta disponibilidade de energia. Devido a esse cenário, será
usada a métrica de probabilidade de perda de oferta (\acrshort{lpsp}) para avaliar as estratégias de \acrlong{lf} e
\acrlong{cc}.  A \acrlong{lpsp}, como citada em~\cite{Upadhyay_2014}, pode ser
descrita pela Equação~\ref{eq:lpsp}, onde DE é o déficit energético e
$P_{\text{load}}$ é a potência demandada pela carga.
O sistema híbrido usado
será constituído de fontes renováveis solar e eólica, banco de baterias e
gerador diesel, visto que esse \acrshort{hres} é amplamente explorado na
literatura.

\begin{equation}
	\text{LPSP} = \frac{\sum_{t=1}^{T} \text{DE}(t)}{\sum_{t=1}^{T} P_{\text{load}}{(t)}\Delta t}
	\label{eq:lpsp}
\end{equation}

Será realizada uma análise preliminar sobre os dados meteorológicos, levando em
consideração a estação meteorológica de Petrolina da Rede \acrshort{sonda}
de~\cite{inpe:sonda}.  Para as previsões, redes neurais recorrentes simples
serão empregadas. Os resultados das previsões serão avaliados para decidir sobre
a estratégia de controle mais adequada, \acrlong{cc} ou \acrlong{lf}.
