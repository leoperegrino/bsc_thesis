% !TEX root = ../00_tcc.tex

\begin{figure}[h]
  \centering

  \scalebox{0.8}{
  \begin{tikzpicture}
    \tikzstyle{neuron}=[circle, draw=black, minimum size = 11mm]
    \tikzstyle{missing}=[draw=none,opacity=0]

	\node[neuron] (out)  at  (4,0)   {$H(\cdot)$};
	\node[neuron] (bias)  at  (4,2)   {$b$};
	\node[]       (resul)  at  (6,0)   {$\hat{y}$};
    \draw[->]     (out)  --  (resul)  ;
    \draw[->]     (bias)  --  (out)  ;

    \foreach \style/\num [count=\i] in {neuron/1,neuron/2,missing/3,neuron/n}{
      \node[\style, draw=none] (input\i)  at  (-1,3-\i*1.5)   {$x_\num$};
	  \node[\style   ]         (nn\i)  at  ( 1,3-\i*1.5)   {$w_\num$};
      \draw[\style,->]         (input\i)  --  (nn\i)  ;
      \draw[\style,->]         (nn\i)  --  (out)  ;
    }

    \node[below of=nn2, node distance=1.5cm] {\LARGE$\vdots$};

  \end{tikzpicture}
}

  \caption{Representação de um Perceptron com $n$ entradas. Fonte: própria.}\label{tikz:perceptron}
\end{figure}
